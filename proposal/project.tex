\documentclass{amsart}
\usepackage{geometry}
\usepackage[utf8]{inputenc}
\usepackage[T1]{fontenc}
\usepackage{graphicx}
\usepackage{float}
\usepackage{tikz}
\usepackage{enumerate}

\geometry{top=0.2in, left=0.2in, right=0.2in, bottom=0.5in}

\author{Navneet Parb, Prajwal V Athreya, Skandan Senthil Nathan}

\title{CS5800 Algorithms - Project}

\begin{document}

\maketitle

\begin{section}{Project Context}

    \begin{subsection}{Navneet}
        \begin{quote}
            ajidnaisdnauinsaiudn    
        \end{quote}
    \end{subsection}

    \begin{subsection}{Prajwal}
        \begin{quote}
            ajidnaisdnauinsaiudn    
        \end{quote}
    \end{subsection}

    \begin{subsection}{Skandan}
        \begin{quote}
            ajidnaisdnauinsaiudn    
        \end{quote}
    \end{subsection}
    
\end{section}

\begin{section}{Objective}
    \begin{quote}
        This project aims to address the challenge of maze generation and solving algorithms in the context of our algorithms course project. Our primary objective is to develop a comprehensive maze-solving simulation that will serve as an invaluable tool for exploring various algorithms, including Breadth-First Search (BFS), Depth-First Search (DFS), A*. By undertaking this project together, we seek to deepen our collective understanding of algorithmic principles and gain practical experience in applying these concepts to solve real-world problems, particularly in maze navigation scenarios.
    \end{quote}
\end{section}

\begin{section}{Scope}
    \begin{quote}
        Within the scope of our project, we plan to develop a maze generation system utilizing Kruskal's algorithm to create solvable mazes of varying complexities. This endeavor will require us to delve into graph theory and data structures to effectively represent the maze environment. Additionally, we will implement multiple pathfinding algorithms to navigate these mazes, considering factors such as optimality, efficiency, and computational complexity. Through our collaborative efforts, we aim to explore different strategies for maze-solving and analyze the strengths and weaknesses of each algorithm. Furthermore, we intend to create a user-friendly interface for visualizing the maze generation process and the execution of pathfinding algorithms, thus facilitating experimentation and learning for both our team and potential users.
    \end{quote}
\end{section}

\begin{section}{Methodology}
    \begin{quote}
        \begin{enumerate}
            \item \textbf{Maze Generation:} We have explored multiple maze generating algorithms, mainly Kruskal's algorithm to generate mazes. This will involve creating a graph representation of the maze and employing disjoint-set data structures to manage the maze's edges and cells. 
            \item \textbf{Pathfinding Algorithms:} We aim to implement Breadth-First Search (BFS), Depth-First Search (DFS), and A* algorithms to solve the generated mazes. We will analyze the performance of these algorithms in terms of path length, computational complexity, and optimality.
        \end{enumerate}
    \end{quote}
\end{section}

\begin{section}{References}

    \begin{quote}
        \begin{enumerate}

        \end{enumerate}
    \end{quote}
    
\end{section}


\end{document}