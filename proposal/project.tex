\documentclass{amsart}
\usepackage{geometry}
\usepackage[utf8]{inputenc}
\usepackage[T1]{fontenc}
\usepackage{graphicx}
\usepackage{float}
\usepackage{tikz}
\usepackage{enumerate}
\usepackage{hyperref}
% \usepackage{enumitem}
\geometry{top=0.2in, left=0.2in, right=0.2in, bottom=0.5in}

\author{Navneet Parab, Prajwal V Athreya, Skandan Senthil Nathan}

\title{CS5800 Algorithms - Project}

\begin{document}

\maketitle

\begin{section}{Project Context}

    \begin{subsection}{Navneet}
        \begin{quote}
            As someone with a background in electronics and telecommunication, the intersection of maze generation and solving using algorithms like Kruskal's and breadth-first search is particularly fascinating to me. Kruskal's algorithm, known for its efficiency in finding minimum spanning trees, holds relevance not only in computer science but also in telecommunications, where it's utilized for network optimization. The application of these algorithms extends beyond traditional computer science domains, playing a crucial role in optimizing routes for signal transmission or data flow in telecommunications networks. Understanding maze generation and solving algorithms not only enhances problem-solving skills but also offers insights into optimizing pathways, which directly translates into improved network efficiency and performance in the telecommunications industry. Exploring these concepts further allows me to bridge my background in electronics with newfound knowledge in computer science, creating innovative solutions for real-world challenges in network optimization. In this group project, my teammates and I will be further exploring maze generation based on Kruskal's algorithm along with solving capabilities based on BFS, DFS and A* algorithms, and discuss the implementation and potential areas of improvement of this topic.
        \end{quote}
    \end{subsection}

    \begin{subsection}{Prajwal}
        \begin{quote}
            I find this project particularly intriguing due to its multifaceted nature, offering a bridge between theoretical knowledge and practical application. Exploring maze generation and solving algorithms provides a unique opportunity to delve into the core principles of algorithms while gaining hands-on experience in their implementation. Developing a maze-solving simulation not only allows me to hone my algorithmic skills but also serves as a creative platform to experiment with diverse strategies and optimizations. The project's dynamic and experiential approach enables me to observe firsthand the real-world implications of various algorithmic techniques. Moreover, navigating through the maze generation process and exploring different pathfinding algorithms will undoubtedly present challenges that will push me to think critically and creatively, ultimately sharpening my problem-solving abilities and deepening my understanding of algorithmic complexities. I have previously worked on autonomous multiagent grid navigating agents using Reinforcement Learning and this project will provide me with a different perspective on how to navigate mazes without iteratively teaching it to an agent.
        \end{quote}
    \end{subsection}

    \begin{subsection}{Skandan}
        \begin{quote}
            My interest in maze generation and pathfinding algorithms stems from my passion for video games. Throughout my journey in this course, I've continuously sought connections between the algorithms we've studied and the mechanics of the games I love. Maze generation and pathfinding stand out as pivotal elements in numerous video game designs.
            
            Maze generation captivates me for its potential to dynamically construct game environments, offering players a fresh challenge with each playthrough. By generating mazes on the fly, games can create suspenseful scenarios where players stealthily navigate through the game world, avoid enemies and reach the exit zone. On the other hand, pathfinding algorithms such as Breadth-First Search (BFS) hold immense significance in video game AI. Implementing BFS and other shortest path algorithms allow game agents to solve mazes and allows non-player characters (NPCs) to follow the player or navigate the game world autonomously. These algorithms are crucial for creating engaging and challenging gameplay experiences.    
        \end{quote}
    \end{subsection}
    
\end{section}

\begin{section}{Objective}
    \begin{quote}
        This project aims to address the challenge of maze generation and solving algorithms in the context of our algorithms course project. Our primary objective is to develop a comprehensive maze-solving simulation that will serve as an invaluable tool for exploring various algorithms, including Breadth-First Search (BFS), Depth-First Search (DFS), A*. By undertaking this project together, we seek to deepen our collective understanding of algorithmic principles and gain practical experience in applying these concepts to solve real-world problems, particularly in maze navigation scenarios.
    \end{quote}
\end{section}

\pagebreak

\begin{section}{Scope}
    \begin{quote}
        Within the scope of our project, we plan to develop a maze generation system utilizing Kruskal's algorithm to create solvable mazes of varying complexities. This endeavor will require us to delve into graph theory and data structures to effectively represent the maze environment. Additionally, we will implement multiple pathfinding algorithms to navigate these mazes, considering factors such as optimality, efficiency, and computational complexity. Through our collaborative efforts, we aim to explore different strategies for maze-solving and analyze the strengths and weaknesses of each algorithm. Furthermore, we intend to create a user-friendly interface for visualizing the maze generation process and the execution of pathfinding algorithms, thus facilitating experimentation and learning for both our team and potential users.
    \end{quote}
\end{section}

\begin{section}{Methodology}
    \begin{quote}
        \begin{enumerate}
            \item \textbf{Maze Generation:} We have explored multiple maze generating algorithms, mainly Kruskal's algorithm to generate mazes. This will involve creating a graph representation of the maze and employing disjoint-set data structures to manage the maze's edges and cells. 
            \item \textbf{Pathfinding Algorithms:} We aim to implement Breadth-First Search (BFS), Depth-First Search (DFS), and A* algorithms to solve the generated mazes. We will analyze the performance of these algorithms in terms of path length, computational complexity, and optimality.
        \end{enumerate}
    \end{quote}
\end{section}

\begin{section}{References}

    \begin{quote}
        \begin{enumerate}
            \item \href{https://www.youtube.com/watch?v=ZMQbHMgK2rw}{Micromouse: The Fastest Maze-Solving Competition On Earth}
            \item \href{https://dahlan.unimal.ac.id/files/ebooks/2009%20Introduction%20to%20Algorithms%20Third%20Ed.pdf}{Cormen, T.H. et al., 2009. Introduction to algorithms, MIT press}
            \item \href{https://www.researchgate.net/profile/Jason-Holdsworth/publication/2727226_The_Nature_of_Breadth-First_Search/links/5539d8d70cf2239f4e7dad9d/The-Nature-of-Breadth-First-Search.pdf}{Holdsworth, Jason. (1999). The Nature of Breadth-First Search. }
            \item \href{https://users.cs.duke.edu/~reif/paper/dfs.ptime.pdf}{Reif, John H. 1983. Depth-first search is inherently sequential, Harvard University}
            \item \href{https://ieeexplore.ieee.org/document/4082128}{Hart, P., Nilsson, N. and Raphael, B. (1968). A Formal Basis for the Heuristic Determination of Minimum Cost Paths. IEEE Transactions on Systems Science and Cybernetics, [online] 4(2), pp.100-107}

        \end{enumerate}
    \end{quote}
    
\end{section}


\end{document}